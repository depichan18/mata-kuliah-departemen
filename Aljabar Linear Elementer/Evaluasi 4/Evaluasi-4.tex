\documentclass[12pt]{article}
\usepackage{amsmath, amssymb}
\usepackage{graphicx}
\usepackage{fancyhdr}
\usepackage{enumitem}
\usepackage{multicol}
\usepackage{geometry}
\geometry{margin=1in}

\pagestyle{fancy}
\fancyhf{}
\rhead{Devi Rosa Aprilla}
\lhead{Assignment}
\cfoot{\thepage}

\begin{document}

\begin{center}
    {Evaluasi Akhir Semester 2025} \\
    \textit{Aljabar Linear Elementer} \\
\end{center}

\vspace{0.5cm}

\begin{enumerate}
    \item Dapatkan determinan dari matriks:
    \[
    A = \begin{bmatrix}
    2 & 1 & 0 & 0 \\
    0 & 0 & 1 & 2 \\
    0 & 1 & -2 & 2 \\
    2 & 3 & 0 & 9
    \end{bmatrix}
    \]

    \item Dengan metode Cramer, selesaikan untuk $y$ dari sistem persamaan berikut:
    \[
    \begin{aligned}
    2x - 2y + z &= 3 \\
    -x + 2y + 3z &= 4 \\
    3x - 2y &= 5
    \end{aligned}
    \]

    \item Diberikan $v$ adalah vektor eigen dari matriks persegi $A$ yang bersesuaian dengan nilai eigen $\lambda$.
    \begin{enumerate}[label=\alph*.]
        \item Tuliskan persamaan yang menghubungkan $A$, $v$, dan $\lambda$.
        \item Selidiki apakah $v$ juga vektor eigen dari $A^k$ untuk $k \geq 2$. Jelaskan jawaban Anda.
    \end{enumerate}

    \item Diberikan matriks:
    \[
    A = \begin{bmatrix}
    2 & 3 \\
    4 & 1
    \end{bmatrix}
    \]
    \begin{enumerate}[label=\alph*.]
        \item Dapatkan nilai eigen dan vektor eigen dari matriks $A$.
        \item Dapatkan matriks $P$ sedemikian hingga $D = P^{-1} A P$ dengan $D$ adalah matriks diagonal.
    \end{enumerate}
\end{enumerate}

\vspace{1cm}
\begin{center}
    \textasciitilde\ Selamat Mengerjakan\textasciitilde
\end{center}

\end{document}
