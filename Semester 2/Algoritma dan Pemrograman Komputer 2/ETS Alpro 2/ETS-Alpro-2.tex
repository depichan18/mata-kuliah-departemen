\documentclass[12pt]{article}
\usepackage{amsmath, amssymb}
\usepackage{graphicx}
\usepackage{fancyhdr}
\usepackage{enumitem}
\usepackage{multicol}
\usepackage{tikz}
\usetikzlibrary{arrows}
\usepackage{tcolorbox}
\usepackage{geometry}
\geometry{margin=1in}

\pagestyle{fancy}
\fancyhf{}
\rhead{Devi Rosa Aprilla}
\lhead{Assignment}
\cfoot{\thepage}

\begin{document}

\begin{center}
    \text{Evaluasi Tengah Semester 2025} \\
    \textit{Algoritma dan Pemrograman Komputer 2} \\
\end{center}

\vspace{0.5cm}

\begin{enumerate}
    \item \textbf{(Kebenaran Algoritma)} \\
    Buktikan kebenaran dari pernyataan berikut:
    \[
        2^n \leq 2^{n+1} - 2^{n-1} - 1, \quad \text{untuk setiap } n \in \mathbb{N}
    \]
    dengan menggunakan induksi matematika.

    \item \textbf{(Analisis Kompleksitas)} \\
    Anda diberi array yang tidak diurutkan berisi $n$ bilangan bulat. Tugas Anda adalah menentukan apakah terdapat triplet $(a, b, c)$ sedemikian hingga:
    \begin{itemize}
        \item $a + b + c = k$, untuk bilangan bulat $k$ yang diberikan.
        \item Ketiga elemen tersebut berbeda (yaitu, nilai yang berbeda, bukan hanya indeks).
        \item Setiap triplet yang valid harus unik, yaitu, permutasi dari angka yang sama tidak dihitung beberapa kali.
    \end{itemize}
    \begin{tcolorbox}[title=Contoh]
    Bil $= [\, 1,\, 2,\, -1,\, -2,\, 0,\, 3,\, -1] \rightarrow n = 7$ \\
    $k = 0$ \\
    Output: \\
    $[\, -2,\, -1,\, 3]$ karena $-2 + (-1) + 3 = 0$, \\
    $[\, -2,\, 0,\, 2]$, $[\,-1,\, -1,\, 2]$, $[\,-1,\, 0,\, 1]$ \\
    \textit{Hint:} left $= i+1$ dan right $= n-1$ dimana $i$ adalah iterasi dari 0 sampai $n-3$.
    \end{tcolorbox}

    \item \textbf{(Searching)} \\
    Buatlah \textit{pseudo code} Java untuk \textbf{mencari angka dalam array yang paling mendekati} angka target yang dimasukkan oleh pengguna.

    \textbf{Ketentuan:}
    \begin{enumerate}
        \item Program menerima input berupa array bilangan bulat.
        \item Pengguna memasukkan satu angka target.
        \item Program mencari angka dalam array yang paling dekat nilainya dengan target.
        \item Tampilkan angka tersebut dan selisihnya dari target.
    \end{enumerate}

        \item \textbf{(Sorting)} \\
    \begin{tcolorbox}[title=Program Java - Sorting]
    \begin{verbatim}
    package ETS;

    public class soal_sorting {
        public static void main(String[] args) {
            int[] nilai = {45, 20, 35, 10, 50, 25};
            ETSSort(nilai);
            for (int i = 0; i < nilai.length; i++) {
                System.out.print(nilai[i] + " ");
            }
        }

        public static void ETSSort(int[] arr) {
            for (int i = 1; i < arr.length; i++) {
                int key = arr[i];
                int j = i - 1;
                while (j >= 0 && arr[j] > key) {
                    arr[j + 1] = arr[j];
                    j--;
                }
                arr[j + 1] = key;
            }
        }
    }
    \end{verbatim}
    \end{tcolorbox}

    \textbf{Pertanyaan:}
    \begin{enumerate}
        \item[a)] Tunjukkan perubahan array \textbf{setelah setiap iterasi luar (loop $i$)} dalam fungsi ETSSort.
        \item[b)] Berapa urutan akhir array yang akan dicetak program?
    \end{enumerate}

    \item \textbf{(Exception Handling)} \\
    \begin{tcolorbox}[title=Program Java - Exception Handling]
    \begin{verbatim}
    public class ExceptionTest {
        public static void main(String[] args) {
            try {
                int[] numbers = {10, 20, 30};
                System.out.println("Nilai pertama: " + numbers[0]);
                int result = numbers[1] / 0; // baris A
                System.out.println("Hasil: " + result);
            } catch (ArithmeticException e) {
                System.out.println("Terjadi kesalahan aritmatika: "
                + e.getMessage());
            } catch (ArrayIndexOutOfBoundsException e) {
                System.out.println("Akses indeks array yang tidak
                valid.");
            } finally {
                System.out.println("Blok finally dijalankan.");
            }
            System.out.println("Program selesai.");
        }
    }
    \end{verbatim}
    \end{tcolorbox}

    \textbf{Pertanyaan:}
    \begin{enumerate}
        \item[a)] Apa yang akan ditampilkan oleh program di atas ketika dijalankan? Jelaskan urutan eksekusi program secara singkat.
        \item[b)] Jika baris \texttt{int result = numbers[1] / 0;} (baris A) diganti menjadi \texttt{int result = numbers[5] / 2;}, output program akan berubah. Jelaskan output yang dihasilkan dan exception mana yang akan ditangkap.
        \item[c)] Apakah blok finally \textbf{selalu} dijalankan? Jelaskan dengan memberikan satu contoh situasi di mana blok finally \textbf{tidak} dijalankan.
    \end{enumerate}

\end{enumerate}

\end{document}