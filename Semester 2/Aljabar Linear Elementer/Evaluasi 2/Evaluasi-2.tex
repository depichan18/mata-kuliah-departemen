\documentclass[12pt]{article}
\usepackage{amsmath, amssymb}
\usepackage{graphicx}
\usepackage{fancyhdr}
\usepackage{enumitem}
\usepackage{multicol}
\usepackage{geometry}
\geometry{margin=1in}

\pagestyle{fancy}
\fancyhf{}
\rhead{Devi Rosa Aprilla}
\lhead{Assignment}
\cfoot{\thepage}

\begin{document}

\begin{center}
    {Evaluasi Tengah Semester 2025} \\
    \textit{Aljabar Linear ELementer} \\
\end{center}

\vspace{0.5cm}

\begin{enumerate}
\item Diberikan vektor $\vec{v}_1 = \begin{bmatrix} 1 \\ 0 \\ -1 \end{bmatrix}$, $\vec{v}_2 = \begin{bmatrix} 1 \\ 2 \\ -1 \end{bmatrix}$, $\vec{v}_1, \vec{v}_2 \in \mathbb{R}^3$.

\begin{enumerate}
\item Tentukan $\vec{v}_1 \times \vec{v}_2$.
\item Dapatkan vektor $\vec{v} \in \mathbb{R}^3$ yang tegak lurus dengan $\vec{v}_1$ dan $\vec{v}_2$.
\item Dapatkan luas jajaran genjang yang ditentukan oleh vektor $\vec{v}_1$ dan $\vec{v}_2$.
\end{enumerate}

\item Diberikan $\vec{u}, \vec{v}, \vec{w} \in \mathbb{R}^4$ dimana
\begin{align*}
\vec{u} = \begin{pmatrix} 1 \\ 2 \\ 0 \\ 0 \end{pmatrix},\, \vec{v} = \begin{pmatrix} 1 \\ 2 \\ 1 \\ 0 \end{pmatrix},\, \vec{w} = \begin{pmatrix} 2 \\ 4 \\ 3 \\ 0 \end{pmatrix}
\end{align*}

\begin{enumerate}
\item Tunjukkan himpunan $S = \{\vec{u}, \vec{v}, \vec{w}\}$ bebas linear.
\item Tambahkan 1 vektor pada himpunan $S$ menjadi himpunan $S^*$ sehingga himpunan $S^*$ merupakan basis dari $\mathbb{R}^4$.
\end{enumerate}

\item Selesaikan sistem persamaan linear berikut dengan metode Gaussian atau Gauss-Jordan
\begin{align*}
x_1 - x_2 + x_3 - 3x_4 &= 0 \\
3x_1 + x_2 - x_3 - x_4 &= 0 \\
2x_1 - x_2 - 2x_3 - x_4 &= 0
\end{align*}

\item Tentukan matriks $A^{-1}$ dari matriks $A$ berikut:
\begin{align*}
A = \begin{bmatrix} 1 & 2 & 3 \\ 0 & -1 & 1 \\ 3 & 5 & 11 \end{bmatrix}
\end{align*}

\end{enumerate}

\vspace{1cm}
\begin{center}
    \textasciitilde\ Selamat Mengerjakan \textasciitilde
\end{center}

\end{document}