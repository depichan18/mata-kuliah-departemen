\documentclass[10pt,a4paper]{article}
\usepackage{graphicx} 
\usepackage{multirow}
\usepackage{enumitem}
\usepackage{amssymb}
\usepackage{amsmath}
\usepackage{amsthm}
\usepackage{xcolor}
\usepackage{array}
\usepackage{geometry}
	\geometry{
		left = 25mm,
		right = 35mm,
		top = 30mm,
		bottom = 30mm
	}
\usepackage{fancyhdr}
\renewcommand{\headrulewidth}{0pt}
\pagestyle{fancy}

\graphicspath{{../../../Insert/}}

\newcommand{\R}{\mathbb{R}}
\newcommand{\N}{\mathbb{N}}
\newcommand{\Z}{\mathbb{Z}}
\newcommand{\Q}{\mathbb{Q}}
\newcommand{\jawab}{\textbf{Solusi}:}

\newtheorem*{teorema}{Teorema}
\newtheorem*{definisi}{Definisi}

\begin{document}
\fancyfoot[C]{\raisebox{.5ex}{\rule{0.5cm}{.4pt}}o0o\raisebox{.5ex}{\rule{0.5cm}{.4pt}}}

\begin{tabular}{@{}r@{}c@{}l@{}}
    \raisebox{0.5cm}{\includegraphics[width=2cm]{ITS.png}}
     &\begin{tabular}{lcll}
        \multicolumn{4}{c}{\begin{tabular}{c}
          \MakeUppercase{evaluasi tengah semester gasal 2024/2025}\\
          \MakeUppercase{departemen matematika fsad its}\\
          \MakeUppercase{program sarjana}\\
     \end{tabular}}\\
     \\
          Matakuliah&:&\multicolumn{2}{l}{Geometri Analitik}\\
          Hari, Tanggal&:&\multicolumn{2}{l}{Rabu, 16 Oktober 2024}\\
          Waktu / Sifat&:&\multicolumn{2}{l}{100 Menit, Tutup buku dan alat-alat digital/elektronik}\\
          Dosen&:&\multicolumn{2}{l}{Drs. I Gst Ngr Rai Usadha, M.Si}\\
          &&\multicolumn{2}{l}{Drs. Is Herisman, M.Si}\\
          &&\multicolumn{2}{l}{Drs. Komar Baihaqi, M.Si}\\
          &&\multicolumn{2}{l}{Dra. Wahyu Doctorina, S.Si, M.Si}\\
     \end{tabular}
     & 
     \raisebox{0.5cm}{\includegraphics[width=2cm]{M.png}}
     \\ \hline
\multicolumn{3}{|p{16cm}|}{\MakeUppercase{harap diperhatikan!!!}}\\
\multicolumn{3}{|p{16cm}|}{Segala jenis pelanggaran (mencontek, kerjasama, dsb) yang dilakukan saat ETS}\\
\multicolumn{3}{|p{16cm}|}{akan dikenakan sanksi pembatalan mata kuliah pada semester yang sedang berjalan.}\\
\hline
\end{tabular}

\vspace{1em}

\begin{enumerate}
    \item \textbf{(20 point)} Diberikan dua lingkaran $L_1: {(x-1)}^2 + {(y-1)}^2 = 5$ dan $L_2: {(x-5)}^2 + {(y-1)}^2 = 13$
    \begin{enumerate}[label=\alph*.]
        \item Tentukan persamaan garis yang melalui titik potong $L_1$ dan $L_2$
        \item Tentukan persamaan lingkaran yang melalui titik potong $L_1$ dan $L_2$ dan titik $P(6,1)$.
    \end{enumerate}
    \item \textbf{(20 point)} Parabola $f_1$ mempunyai titik fokus $F(2,3)$ dan direktris $y=2$. Sedangkan parabola $f_2$ dengan titik fokus $F'(-1,-4)$ dan direktris $y = 1$.
    \begin{enumerate}[label=\alph*.]
        \item Tentukan persamaan parabola $f_1$ dan $f_2$.
        \item Tentukan persamaan garis singgung $L_1$ dari $f_1$ yang melalui titik $Q(1,3)$.
        \item Jika titik $A(x_1, y_1)$ pada $f_2$ sedemikian sehingga garis singgung $L_2$ dari $f_2$ di $A(x_1, y_1)$ tegak lurus dengan $L_1$.
    \end{enumerate}
    \item \textbf{(20 point)} Tentukan persamaan parabola, dengan garis direktris $y=1-x$ dan fokus titik $F(0,2)$, serta sketsa grafiknya. (Catatan: Turunkan persamaannya dari definisi parabola.)
    \item \textbf{(20 point)} Tentukan matriks Transformasi Linear dari $T\left(\begin{bmatrix}x_1\\x_2\end{bmatrix}\right) = \begin{bmatrix}x_1 + x_2 \\ x_1 - x_2 \\ 2x_1 + 3x_2\end{bmatrix}$
    \item \textbf{(20 point)} Tentukan persamaan ellips, jika diketahui: Titik fokusnya $(\pm 3, 0)$ dan eksentrisitasnya, $e = \frac{3}{5}$.
\end{enumerate}

\vspace{2em}
\begin{center}
    \textbf{***** Selamat mengerjakan semoga sukses *****}\\[0.5em]
    \textit{Kejujuran adalah modal utama suatu keberhasilan}
\end{center}

\end{document}