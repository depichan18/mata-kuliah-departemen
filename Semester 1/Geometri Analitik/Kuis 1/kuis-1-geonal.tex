\documentclass[10pt,a4paper]{article}
\usepackage{graphicx} 
\usepackage{multirow}
\usepackage{enumitem}
\usepackage{amssymb}
\usepackage{amsmath}
\usepackage{amsthm}
\usepackage{xcolor}
\usepackage{geometry}
	\geometry{
		left = 25mm,
		right = 35mm,
		top = 30mm,
		bottom = 30mm,
	}
\usepackage{fancyhdr}
\renewcommand{\headrulewidth}{0pt}
\pagestyle{fancy}

\graphicspath{{../../../Insert/}}

\newcommand{\R}{\mathbb{R}}
\newcommand{\N}{\mathbb{N}}
\newcommand{\Z}{\mathbb{Z}}
\newcommand{\Q}{\mathbb{Q}}
\newcommand{\jawab}{\textbf{Solusi}:}

\newtheorem*{teorema}{Teorema}
\newtheorem*{definisi}{Definisi}

\begin{document}
\fancyfoot[C]{\raisebox{.5ex}{\rule{0.5cm}{.4pt}}o0o\raisebox{.5ex}{\rule{0.5cm}{.4pt}}}

\begin{tabular}{r c l}
    \includegraphics[width=2cm]{ITS.png}
     &\begin{tabular}{lcll}
        \multicolumn{4}{c}{\begin{tabular}{c}
          \MakeUppercase{quiz 1 semester gasal 2024/2025}\\
          \MakeUppercase{departemen matematika fsad its}\\
          \MakeUppercase{program sarjana}\\
     \end{tabular}}\\
     \\
          Matakuliah&:&\multicolumn{2}{l}{Geometri Analitik}\\
          Hari, Tanggal&:&\multicolumn{2}{l}{Rabu, 11 September 2025}\\
          Waktu / Sifat&:&\multicolumn{2}{l}{90 menit / tertutup}\\
          Dosen&:&\multicolumn{2}{l}{Drs. Komar Baihaqi, M.Si}\\
     \end{tabular}
     & 
     \includegraphics[width=2cm]{M.png}
     \\ \hline
\multicolumn{3}{|l|}{\MakeUppercase{harap diperhatikan !!!}}\\
\multicolumn{3}{|l|}{Segala jenis pelanggaran (mencontek, kerjasama, dsb) yang dilakukan pada saat Quiz}\\
\multicolumn{3}{|l|}{akan dikenakan sanksi pembatalan matakuliah pada semester yang sedang berjalan.}\\
\hline
\end{tabular}

\vspace{1em}

\begin{enumerate}
    \item Tentukan persamaan garis yang berjarak 7 unit dari pusat dan bergradien 3.
    \item Tentukan persamaan garis yang melalui $4x + y - 11 = 0$ dan $3x - y = 3$ dan tegak lurus dengan $x + 10y = 7$.
    \item Tentukan persamaan lingkaran melalui titik $(2,-1)$, garis singgung lingkaran $x + y = 1$ dan pusat lingkaran melalui $y = -2x$.
    \item Tentukan persamaan lingkaran yang melalui pada lingkaran $x^2 + y^2 - x + 2y = 3$ dan $x^2 + y^2 - 6x = 4$ melalui titik $(2, -1)$.
    \item Tentukan persamaan lingkaran yang memuat $(2,-1)$ dan titik perpotongan
    \[
    x^2 + y^2 - 2x - 4y + 1 = 0 \quad \text{dan} \quad x^2 + y^2 - 6x - 2y + 9 = 0
    \]
    serta sketlah grafiknya.
\end{enumerate}

\vspace{2em}
\begin{center}
    \textbf{***** Selamat mengerjakan semoga sukses *****} \\
    \textit{Kejujuran adalah modal utama suatu keberhasilan}
\end{center}

\end{document}