% Format sesuai permintaan
\documentclass[10pt,a4paper]{article}
\usepackage{graphicx} 
\usepackage{multirow}
\usepackage{enumitem}
\usepackage{amssymb}
\usepackage{amsmath}
\usepackage{amsthm}
\usepackage{xcolor}
\usepackage{array}
\usepackage{geometry}
    \geometry{
        left = 25mm,
        right = 35mm,
        top = 30mm,
        bottom = 30mm
    }
\usepackage{fancyhdr}
\renewcommand{\headrulewidth}{0pt}
\pagestyle{fancy}

\graphicspath{{../../../Insert/}}

\newcommand{\R}{\mathbb{R}}
\newcommand{\N}{\mathbb{N}}
\newcommand{\Z}{\mathbb{Z}}
\newcommand{\Q}{\mathbb{Q}}
\newcommand{\jawab}{\textbf{Solusi}:}

\newtheorem*{teorema}{Teorema}
\newtheorem*{definisi}{Definisi}

\begin{document}
% Footer
\fancyfoot[C]{\raisebox{.5ex}{\rule{0.5cm}{.4pt}}o0o\raisebox{.5ex}{\rule{0.5cm}{.4pt}}}

% Header Tabel
\begin{tabular}{@{}r@{}c@{}l@{}}
    \raisebox{0.5cm}{\includegraphics[width=2cm]{ITS.png}}
     &\begin{tabular}{lcll}
        \multicolumn{4}{c}{\hspace*{-1cm}\begin{tabular}{c}
          \MakeUppercase{evaluasi akhir semester gasal 2024/2025}\\
          \MakeUppercase{departemen matematika fsad its}\\
          \MakeUppercase{program studi sarjana}\\
     \end{tabular}}\\
     \\
          Matakuliah&:&\multicolumn{2}{l}{Geometri Analitik}\\
          Hari, Tanggal&:&\multicolumn{2}{l}{Selasa, 10 Desember 2024}\\
          Waktu / Sifat&:&\multicolumn{2}{l}{11.00-12.40 / \textit{Closed Book}}\\
          Kelas, Dosen&:&A.& M. Syifa’ul Mufid, D.Phil. \& I Gst Ngr Rai Usadha, M.Si\\
          &&B.& Drs. Isis Herisma, M.Sc.\\
          &&C.& Drs. Komar Baihaqi, M.Si.\\
          &&D.& Dr.mont. Kistosoji Fahim \& Wahyu Fistia D., M.Si.\\
     \end{tabular}
     & 
     \hspace{-1cm}\raisebox{0.5cm}{\includegraphics[width=2cm]{M.png}}
     \\ \hline
\multicolumn{3}{|p{16cm}|}{\MakeUppercase{harap diperhatikan!!!}}\\
\multicolumn{3}{|p{16cm}|}{Segala jenis pelanggaran (mencontek, kerjasama, dsb) yang dilakukan saat ETS}\\
\multicolumn{3}{|p{16cm}|}{akan dikenakan sanksi pembatalan mata kuliah pada semester yang sedang berjalan.}\\
\hline
\end{tabular}

% Spasi sebelum soal
\vspace{1em}

% Soal

\begin{enumerate}
    \item Tentukan persamaan garis singgung hiperbola $\dfrac{x^2}{64} - \dfrac{y^2}{36} = 1$ yang tegak lurus dengan garis $x - 2y + 3 = 0$.
    \item Dapatkan nilai eksentrisitas, koordinat titik fokus, dan persamaan direktris dari irisan kerucut yang diberikan dengan persamaan: $7x^2 + 13y^2 + 6\sqrt{3}xy = 64$. Kemudian sketsa kurvanya.
    \item Diberikan persamaan kutub $r = 6$ dan $r = 4 + 4\cos(\theta)$.
    \begin{enumerate}[label=(\alph*)]
        \item Sketsa persamaan tersebut dalam satu bidang koordinat kutub.
        \item Tentukan titik potong kedua persamaan tersebut.
    \end{enumerate}
    \item Temukan bentuk umum persamaan bidang yang melalui titik $(2,1,1)$, $(0,4,1)$, dan $(-2,1,4)$.
    \item Sketsa trace $xy$, trace $yz$, dan trace $zx$ dari permukaan $x^2 - 4y^2 + z^2 = 36$.
\end{enumerate}

% Penutup
\vspace{2em}
\begin{center}
    \textbf{***** Selamat mengerjakan semoga sukses *****}\\[0.5em]
    \textit{Kejujuran adalah modal utama suatu keberhasilan}
\end{center}

\end{document}