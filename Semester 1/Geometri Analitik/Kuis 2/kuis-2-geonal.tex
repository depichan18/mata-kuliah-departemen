\documentclass[10pt,openany,a4paper]{article}
\usepackage{graphicx} 
\usepackage{multirow}
\usepackage{enumitem}
\usepackage{amssymb}
\usepackage{amsmath}
\usepackage{amsthm}
\usepackage{xcolor}
\usepackage{geometry}
	\geometry{
		total = {160mm, 237mm},
		left = 25mm,
		right = 35mm,
		top = 30mm,
		bottom = 30mm,
	}
\usepackage{fancyhdr}
\renewcommand{\headrulewidth}{0pt}
\pagestyle{fancy}

\graphicspath{{../../../Insert/}}

\newcommand{\R}{\mathbb{R}}
\newcommand{\N}{\mathbb{N}}
\newcommand{\Z}{\mathbb{Z}}
\newcommand{\Q}{\mathbb{Q}}
\newcommand{\jawab}{\textbf{Solusi}:}

\newtheorem*{teorema}{Teorema}
\newtheorem*{definisi}{Definisi}

\begin{document}
\fancyfoot[C]{\raisebox{.5ex}{\rule{0.5cm}{.4pt}}o0o\raisebox{.5ex}{\rule{0.5cm}{.4pt}}}

\begin{tabular}{r c l}
    \includegraphics[width=2cm]{ITS.png}
     &\begin{tabular}{lcll}
        \multicolumn{4}{c}{\begin{tabular}{c}
          \MakeUppercase{quiz 2 semester gasal 2024/2025}\\
          \MakeUppercase{departemen matematika fsad its}\\
          \MakeUppercase{program sarjana}\\
     \end{tabular}}\\
     \\
          Matakuliah&:&\multicolumn{2}{l}{Geometri Analitik}\\
          Hari, Tanggal&:&\multicolumn{2}{l}{Senin, 25 November 2024}\\
          Waktu / Sifat&:&\multicolumn{2}{l}{60 menit / mandiri}\\
          Dosen&:&\multicolumn{2}{l}{Drs. Komar Baihaqi, M.Si}\\
     \end{tabular}
     & 
     \includegraphics[width=2cm]{M.png}
     \vspace{0.5cm}
     \\ \hline
\multicolumn{3}{|l|}{\textbf{HARAP DIPERHATIKAN !!!}}\\
\multicolumn{3}{|l|}{Segala jenis pelanggaran (mencontek, kerjasama, dsb) yang dilakukan saat ETS/EAS akan dikenakan}\\
\multicolumn{3}{|l|}{sanksi pembatalan mata kuliah pada semester yang sedang berjalan.}\\
\hline
\end{tabular}

\vspace{0.5cm}

\begin{enumerate}
    \item Tentukan titik pusat, titik-titik ujung (vertex) dan titik-titik fokus dari persamaan Hiperbola $6x^2 - 9y^2 + 32x + 36y - 164 = 0$
    \item Tentukan persamaan garis singgung hiperbola $\dfrac{x^2}{64} - \dfrac{y^2}{36} = 1$, yang tegak lurus dengan garis $x - 2y + 3 = 0$
    \item Tentukan titik pusat, persamaan direktris, eksentris dari persamaan $r = \dfrac{15}{2-3\cos\theta}$
    \item Tentukan pusat dan radius pada bola \\
    $x^2 + y^2 + z^2 + 6x - 8y + 2z = 10$
\end{enumerate}

\vspace{0.5cm}

\begin{center}
    \textbf{***** Selamat mengerjakan semoga sukses *****}\\
    \textit{Kejujuran adalah modal utama suatu keberhasilan}
\end{center}

\end{document}
